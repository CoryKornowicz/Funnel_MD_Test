%%%%%%%%%%%%%%%%%%%%%%%%%%%%%%%%%%%%%%%%%%%%%%%%%%%%%%%%%%%%
%%% LIVECOMS ARTICLE TEMPLATE FOR TRAINING ARTICLE
%%% ADAPTED FROM ELIFE ARTICLE TEMPLATE (8/10/2017)
%%%%%%%%%%%%%%%%%%%%%%%%%%%%%%%%%%%%%%%%%%%%%%%%%%%%%%%%%%%%
%%% PREAMBLE
\documentclass[9pt,training]{livecoms}
% Use the 'onehalfspacing' option for 1.5 line spacing
% Use the 'doublespacing' option for 2.0 line spacing
% Use the 'lineno' option for adding line numbers.
% Use the "ASAPversion' option following article acceptance to add the DOI and relevant dates to the document footer.
% Use the 'pubversion' option for adding the citation and publication information to the document footer, when the LiveCoMS issue is finalized.
% The 'training' option for indicates that this is a training article.
% Omit the bestpractices option to remove the marking as a LiveCoMS paper.
% Please note that these options may affect formatting.

\usepackage{lipsum} % Required to insert dummy text
\usepackage[version=4]{mhchem}
\usepackage{siunitx}
\DeclareSIUnit\Molar{M}
\usepackage[italic]{mathastext}
\graphicspath{{figures/}}

%%%%%%%% USER PACKAGES
\usepackage{markdown}
\usepackage{minted}
\usepackage{tcolorbox}
\tcbuselibrary{minted}
\usepackage{listings}
\lstset{
	basicstyle=\ttfamily,
	commentstyle={},
	breakatwhitespace=true,
	breaklines=true,
	language=bash
}
%%%%%%%%%%%%%%%%%%%%%%%%%%%%%%%%%%%%%%%%%%%%%%%%%%%%%%%%%%%%
%%% IMPORTANT USER CONFIGURATION
%%%%%%%%%%%%%%%%%%%%%%%%%%%%%%%%%%%%%%%%%%%%%%%%%%%%%%%%%%%%

\newcommand{\versionnumber}{1.3}  % you should update the minor version number in preprints and major version number of submissions.
\newcommand{\githubrepository}{\url{https://github.com/myaccount/homegithubrepository}}  %this should be the main github repository for this article

%%%%%%%%%%%%%%%%%%%%%%%%%%%%%%%%%%%%%%%%%%%%%%%%%%%%%%%%%%%%
%%% ARTICLE SETUP
%%%%%%%%%%%%%%%%%%%%%%%%%%%%%%%%%%%%%%%%%%%%%%%%%%%%%%%%%%%%
\title{This is the title [Article v\versionnumber]}

\author[1*]{Firstname Middlename Surname}
\author[1,2\authfn{1}\authfn{3}]{Firstname Middlename Familyname}
\author[2\authfn{1}\authfn{4}]{Firstname Initials Surname}
\author[2*]{Firstname Surname}
\affil[1]{Institution 1}
\affil[2]{Institution 2}

\corr{email1@example.com}{FMS}  % Correspondence emails.  FMS and FS are the appropriate authors initials.
\corr{email2@example.com}{FS}

\orcid{Author 1 name}{AAAA-BBBB-CCCC-DDDD}
\orcid{Author 2 name}{EEEE-FFFF-GGGG-HHHH}

\contrib[\authfn{1}]{These authors contributed equally to this work}
\contrib[\authfn{2}]{These authors also contributed equally to this work}

\presentadd[\authfn{3}]{Department, Institute, Country}
\presentadd[\authfn{4}]{Department, Institute, Country}

\blurb{This LiveCoMS document is maintained online on GitHub at \githubrepository; to provide feedback, suggestions, or help improve it, please visit the GitHub repository and participate via the issue tracker.}

%%%%%%%%%%%%%%%%%%%%%%%%%%%%%%%%%%%%%%%%%%%%%%%%%%%%%%%%%%%%
%%% PUBLICATION INFORMATION
%%% Fill out these parameters when available
%%% These are used when the "pubversion" option is invoked
%%%%%%%%%%%%%%%%%%%%%%%%%%%%%%%%%%%%%%%%%%%%%%%%%%%%%%%%%%%%
\pubDOI{10.XXXX/YYYYYYY}
\pubvolume{<volume>}
\pubissue{<issue>}
\pubyear{<year>}
\articlenum{<number>}
\datereceived{Day Month Year}
\dateaccepted{Day Month Year}

%%%%%%%%%%%%%%%%%%%%%%%%%%%%%%%%%%%%%%%%%%%%%%%%%%%%%%%%%%%%
%%% ARTICLE START
%%%%%%%%%%%%%%%%%%%%%%%%%%%%%%%%%%%%%%%%%%%%%%%%%%%%%%%%%%%%

\begin{document}

\begin{frontmatter}
\maketitle

\begin{abstract}
This particular document provides a skeleton illustrating key sections for a Training article.
Please see the sample \texttt{sample-document.tex} in \url{github.com/livecomsjournal/article_templates/templates} for additional information on and examples of using the LiveCoMS LaTeX class.
Here we also assume familiarity with LaTeX and knowledge of how to include figures, tables, etc.; if you want examples, see the sample just referenced.

In your work, in this particular slot, please provide an abstract of no more than 250 words.
Your abstract should explain the main contributions of your article, and should not contain any material that is not included in the main text.
Please note that your abstract, plus the authorship material following it, must not extend beyond the title page or modifications to the LaTeX class will likely be needed.
\end{abstract}

\end{frontmatter}




\section{Introduction}

Here you would explain what problem you are tackling and briefly motivate your work.

In this particular template, we have removed most of the usage examples which occur in \texttt{sample-document.tex} to provide a minimal template you can modify; however, we retain a couple of examples illustrating more unusual features of our templates/article class, such as the checklists, and information on algorithms and pseudocode.

Keep in mind, as you prepare your manuscript, that you should plan for a representative image  which will be used to highlight your article on the journal website and publications. Usually, this would be one of your figures, but it must also be uploaded separately upon article submission. We give specific guidelines for this image on the journal website in the section on article submission (see \url{https://livecomsjournal.github.io/authors/policies/index.html#article-submission}).

Additionally, for well-formatted manuscripts, we recommend that you let LaTeX handle figure/table placement for you as much as possible, so please avoid specifying strenuous float instructions like `[h!]` and `[H]` as much as possible.

\subsection{Scope}

Training articles are largely self-contained and introduce readers to the theory and/or typical procedures for a particular aspect of molecular simulation.

\begin{lstlisting}
$ command -arg1 -arg2 -arg3 -arg4 -arg5 -arg6 > output.file 2> err.file
\end{lstlisting}
monokai

\begin{tcblisting}{size=fbox,
minted language=python}
import numpy as np
    
def incmatrix(genl1,genl2):
    m = len(genl1)
    n = len(genl2)
    M = None #to become the incidence matrix
    VT = np.zeros((n*m,1), int)  #dummy variable

    return M

\end{tcblisting}

Training articles need not refer to specific software commands and procedures but may do so to further the pedagogical goals of the paper.

As needed, training articles are likely to display more details of calculations than typically are shown in research articles, for example, noting any mathematical principles or “tricks” used to derive key results.

The scope of the training article should be clearly defined.
This will often happen in a specific section or subsection in the article itself.

\section{Prerequisites}

Training articles should clearly state the target audience and knowledge prerequisites.
Key prerequisites should be noted in the article abstract to permit readers to rapidly ascertain an articles suitability.

\subsection{Background knowledge}

Although the authors may imagine a particular career-level (e.g., undergraduate or graduate), given the diversity of disciplinary curricula, it is more important to specify precisely any knowledge prerequisites (e.g., vector calculus, basic thermodynamics).

\subsection{Software/system requirements}

If a particular software or programming environment plays a central role in the article, that should be specified.

\section{Content and links}

A training article may on additional files and materials; clearly indicate where and how these are available, with links, and how they are being archived for the long-term and maintained so they stay current.
You will likely want to reference your GitHub repository as a central point to access all of this information, and then the GitHub repository may link out to other content as needed.

%%%%%%%% TUTORIAL SECTIONS

\subsection{Introduction to BioSimSpace}

\markdownInput{01_introduction/01_introduction.md}


\subsection{Funnel Meta-Molecular Dynamics}
\newenvironment{cols}[1][]{}{}

\newenvironment{col}[1]{\begin{minipage}{#1}\ignorespaces}{%
\end{minipage}
\ifhmode\unskip\fi
\aftergroup\useignorespacesandallpars}

\def\useignorespacesandallpars#1\ignorespaces\fi{%
#1\fi\ignorespacesandallpars}

\makeatletter
\def\ignorespacesandallpars{%
  \@ifnextchar\par
    {\expandafter\ignorespacesandallpars\@gobble}%
    {}%
}
\makeatother
Let's import the BioSimSpace Python package and see what we can do. For
convenience, we'll rename the package to BSS to save us typing:

\begin{Shaded}
\begin{Highlighting}[]
\ImportTok{import}\NormalTok{ BioSimSpace }\ImportTok{as}\NormalTok{ BSS}
\end{Highlighting}
\end{Shaded}

To see what file formats are supported by BioSimSpace, execute the cell
below.

\begin{Shaded}
\begin{Highlighting}[]
\NormalTok{BSS.IO.fileFormats()}
\end{Highlighting}
\end{Shaded}

\begin{verbatim}
['Gro87', 'GroTop', 'MOL2', 'PDB', 'PRM7', 'PSF', 'RST', 'RST7']
\end{verbatim}

Note that these refer to specific file \emph{formats}, rather than file
\emph{extensions}. BioSimSpace doesn't care about file extensions, it's
the \emph{contents} of the file that's important.

If you aren't familiar with a particular format, you can get more
information as follows, e.g.:

\begin{Shaded}
\begin{Highlighting}[]
\NormalTok{BSS.IO.formatInfo(}\StringTok{"GroTop"}\NormalTok{)}
\end{Highlighting}
\end{Shaded}

\begin{verbatim}
'Gromacs Topology format files.'
\end{verbatim}

The \texttt{BSS.IO.readMolecules} function is used to read molecular
information from file. We've provided some example input files for you
in the \texttt{inputs} directory. Let's take a look at some of these.

\begin{Shaded}
\begin{Highlighting}[]
\OperatorTok{!}\NormalTok{ls inputs}
\end{Highlighting}
\end{Shaded}

\begin{verbatim}
1jr5.crd  1jr5.top  ala.crd  kigaki.gro  methanol.pdb
1jr5.pdb  2JJC.pdb  ala.top  kigaki.top
\end{verbatim}

The \texttt{ala.crd} and \texttt{ala.top} files define a solvated
alanine dipeptide system in AMBER format. Execute the cell below to see
part of the topology file:

\begin{Shaded}
\begin{Highlighting}[]
\OperatorTok{!}\NormalTok{head }\OperatorTok{{-}}\NormalTok{n }\DecValTok{20}\NormalTok{ inputs}\OperatorTok{/}\NormalTok{ala.top}
\end{Highlighting}
\end{Shaded}

{\footnotesize
\begin{lstlisting}[columns=flexible]

%VERSION  VERSION_STAMP = V0001.000  DATE = 06/30/15  11:44:23                  
%FLAG TITLE                                                                     
%FORMAT(20a4)                                                                   
ACE                                                                             
%FLAG POINTERS                                                                  
%FORMAT(10I8)                                                                   
    1912       9    1902       9      25      11      43      24       0       0
    2619     633       9      11      24      13      21      20      10       1
       0       0       0       0       0       0       0       1      10       0
       0
%FLAG ATOM_NAME                                                                 
%FORMAT(20a4)                                                                   
HH31CH3 HH32HH33C   O   N   H   CA  HA  CB  HB1 HB2 HB3 C   O   N   H   CH3 HH31
HH32HH33O   H1  H2  O   H1  H2  O   H1  H2  O   H1  H2  O   H1  H2  O   H1  H2  
O   H1  H2  O   H1  H2  O   H1  H2  O   H1  H2  O   H1  H2  O   H1  H2  O   H1  
H2  O   H1  H2  O   H1  H2  O   H1  H2  O   H1  H2  O   H1  H2  O   H1  H2  O   
H1  H2  O   H1  H2  O   H1  H2  O   H1  H2  O   H1  H2  O   H1  H2  O   H1  H2  
O   H1  H2  O   H1  H2  O   H1  H2  O   H1  H2  O   H1  H2  O   H1  H2  O   H1  
H2  O   H1  H2  O   H1  H2  O   H1  H2  O   H1  H2  O   H1  H2  O   H1  H2  O   
H1  H2  O   H1  H2  O   H1  H2  O   H1  H2  O   H1  H2  O   H1  H2  O   H1  H2 
\end{lstlisting}
}
\begin{lstlisting}[columns=flexible]

DISTANCERESSPEC
# DISH  DISC
  0.1   0.153
# i     j  k  l  type i    j  k  l  type r0     w0    rah
  1208  0  0  0  0    309  0  0  0  0    0.22   1.0   0
  1208  0  0  0  0    321  0  0  0  0    0.235  1.0   0
  1208  0  0  0  0    33   0  0  0  0    0.246  1.0   0
  120   0  0  0  0    489  0  0  0  0    0.255  1.0   0
  1208  0  0  0  0    490  0  0  0  0    0.248  1.0   0
END
\end{lstlisting}

\subsection{Steered Dynamics}

\subsection{Free Energy Perturbation}


\section{Checklists}
Training articles do not necessarily require the use of a checklist as in Best Practices documents; however, they can include these if desired.
Several useful checklist formats are available, with examples presented in \texttt{sample-document.tex} in \url{github.com/livecomsjournal/article_templates/templates}.
One example is shown here.

% Here is a single-column checklist that consists of multiple sub-checklists
\begin{Checklists}

\begin{checklist}{A list}
\textbf{Single-column checklists are also straightforward by removing the asterisk}
\begin{itemize}
\item First thing let's do an item which breaks across lines to see how that looks
\item Also remember
\item And finally
\end{itemize}
\end{checklist}

\begin{checklist}{Another list}
\textbf{This is some further description.}
\begin{itemize}
\item First thing
\item Also remember
\item And finally
\end{itemize}
\end{checklist}

\end{Checklists}








\section{Author Contributions}
%%%%%%%%%%%%%%%%
% This section mustt describe the actual contributions of
% author. Since this is an electronic-only journal, there is
% no length limit when you describe the authors' contributions,
% so we recommend describing what they actually did rather than
% simply categorizing them in a small number of
% predefined roles as might be done in other journals.
%
% See the policies ``Policies on Authorship'' section of https://livecoms.github.io
% for more information on deciding on authorship and author order.
%%%%%%%%%%%%%%%%

(Explain the contributions of the different authors here)

% We suggest you preserve this comment:
For a more detailed description of author contributions,
see the GitHub issue tracking and changelog at \githubrepository.

\section{Other Contributions}
%%%%%%%%%%%%%%%
% You should include all people who have filed issues that were
% accepted into the paper, or that upon discussion altered what was in the paper.
% Multiple significant contributions might mean that the contributor
% should be moved to authorship at the discretion of the a
%
% See the policies ``Policies on Authorship'' section of https://livecoms.github.io for
% more information on deciding on authorship and author order.
%%%%%%%%%%%%%%%

(Explain the contributions of any non-author contributors here)
% We suggest you preserve this comment:
For a more detailed description of contributions from the community and others, see the GitHub issue tracking and changelog at \githubrepository.

\section{Potentially Conflicting Interests}
%%%%%%%
%Declare any potentially competing interests, financial or otherwise
%%%%%%%

Declare any potentially conflicting interests here, whether or not they pose an actual conflict in your view.

\section{Funding Information}
%%%%%%%
% Authors should acknowledge funding sources here. Reference specific grants.
%%%%%%%
FMS acknowledges the support of NSF grant CHE-1111111.

\section*{Author Information}
\makeorcid

\bibliography{livecoms-sample}

%%%%%%%%%%%%%%%%%%%%%%%%%%%%%%%%%%%%%%%%%%%%%%%%%%%%%%%%%%%%
%%% APPENDICES
%%%%%%%%%%%%%%%%%%%%%%%%%%%%%%%%%%%%%%%%%%%%%%%%%%%%%%%%%%%%

%\appendix


\end{document}