Author: Lester Hedges Email:~~ lester.hedges@bristol.ac.uk

\hypertarget{running-nodes}{%
\section{Running nodes}\label{running-nodes}}

The companion notebook for this section can be found
\href{https://github.com/michellab/BioSimSpaceTutorials/blob/4844562e7d2cd0b269cead56562ec16a3dfaef7c/01_introduction/05_running_nodes.ipynb}{here}

The previous section showed you how to write a node to perform
minimisation of a molecular system within an interactive Jupyter
notebook. Here we introduce you to some of the other ways of running
BioSimSpace nodes, showing how the same script can be used in several
different ways.

\hypertarget{running-nodes-on-the-command-line}{%
\subsection{Running nodes on the
command-line}\label{running-nodes-on-the-command-line}}

The typical way of interacting with BioSimSpace is by running a workflow
component, or \emph{node}, from the command-line. A node is just a
normal Python script that is run using the \texttt{python} interpreter.
Let's use the molecular minimisation example from the previous notebook,
which we've provided as a Python script called \texttt{minimisation.py}
within \texttt{nodes} directory. (This is the just the previous
notebook, downloaded as a regular Python script.)

From the command-line, we can query the node to see what it does and get
information about the inputs:

\begin{Shaded}
\begin{Highlighting}[]
\NormalTok{python nodes}\OperatorTok{/}\NormalTok{minimise.py }\OperatorTok{--}\BuiltInTok{help}
\end{Highlighting}
\end{Shaded}

\begin{verbatim}
usage: minimise.py [-h] [-c CONFIG] [-v [VERBOSE]] [--export-cwl [EXPORT_CWL]]
                   [--strict-file-naming [STRICT_FILE_NAMING]] --files FILES
                   [FILES ...] [--steps STEPS]
                   [--engine {Amber,Gromacs,Namd,OpenMM,Somd,auto}]

A node to perform energy minimisation and save the minimised molecular
configuration to file.

Args that start with '--' (e.g. --arg) can also be set in a config file
(specified via -c). The config file uses YAML syntax and must represent a YAML
'mapping' (for details, see http://learn.getgrav.org/advanced/yaml). If an arg
is specified in more than one place, then commandline values override config
file values which override defaults.

Output:
  minimised: FileSet    The minimised molecular system.

Required arguments:
  --files FILES [FILES ...]
                        A set of molecular input files.

Optional arguments:
  -h, --help            Show this help message and exit.
  -c CONFIG, --config CONFIG
                        Path to configuration file.
  -v [VERBOSE], --verbose [VERBOSE]
                        Print verbose error messages.
  --export-cwl [EXPORT_CWL]
                        Export Common Workflow Language (CWL) wrapper and exit.
  --strict-file-naming [STRICT_FILE_NAMING]
                        Enforce that the prefix of any file based output matches its name.
  --steps STEPS         The number of minimisation steps.
                          default=10000
                          min=0, max=1000000
  --engine {Amber,Gromacs,Namd,OpenMM,Somd,auto}
                        The molecular dynamics engine.
                          default=auto
\end{verbatim}

In the previous section, input was achieved via a graphical user
interface where the user could configure options and upload files. On
the command-line, inputs must be set as command-line arguments. From the
information provided in the node itself, i.e.~the description, the
definition of inputs and outputs, BioSimSpace has autogenerated a nicely
formatted
\href{https://docs.python.org/3/library/argparse.html}{argparse} help
message that describes how the node works. The information shows all of
the inputs and outputs, let's us know which inputs are optional, and
specifies any default values or constraints.

Note that it's possible to pass options to the node in various ways,
e.g.~directly on the command-line, using a
\href{https://en.wikipedia.org/wiki/YAML}{YAML} configuration file, or
even using environment variables. This provides a lot of flexibility in
the way in which BioSimSpace nodes can be run. For now we'll just pass
arguments on the command-line.

Try running the node without any arguments and seeing what the output
is:

\begin{Shaded}
\begin{Highlighting}[]
\NormalTok{python nodes}\OperatorTok{/}\NormalTok{minimise.py}
\end{Highlighting}
\end{Shaded}

\begin{verbatim}
usage: minimise.py [-h] [-c CONFIG] [-v [VERBOSE]] [--export-cwl [EXPORT_CWL]]
                   [--strict-file-naming [STRICT_FILE_NAMING]] --files FILES
                   [FILES ...] [--steps STEPS]
                   [--engine {Amber,Gromacs,Namd,OpenMM,Somd,auto}]
minimise.py: error: the following arguments are required: --files
\end{verbatim}

Thankfully we've provided some files for you. As before, these are found
in the \texttt{input} directory.

\begin{Shaded}
\begin{Highlighting}[]
\NormalTok{ls inputs}\OperatorTok{/}\NormalTok{ala}\OperatorTok{*}
\end{Highlighting}
\end{Shaded}

\begin{verbatim}
inputs/ala.crd  inputs/ala.top
\end{verbatim}

(The files define a solvated alanine dipeptide system in
\href{http://ambermd.org}{AMBER} format.)

Let's now run the minimisation node using these files as input. In the
interests of time, let's also reduce the number of steps to 1000. The
files can be passed to the script in various ways. All of the following
are allowed:

\begin{Shaded}
\begin{Highlighting}[]
\ExtensionTok{python}\NormalTok{ nodes/minimisation.py --steps=1000 --files=}\StringTok{"inputs/ala.crd, inputs/ala.top"}
\ExtensionTok{python}\NormalTok{ nodes/minimisation.py --steps=1000 --files inputs/ala.crd inputs/ala.top}
\ExtensionTok{python}\NormalTok{ nodes/minimisation.py --steps=1000 --files inputs/ala.*}
\end{Highlighting}
\end{Shaded}

\begin{Shaded}
\begin{Highlighting}[]
\NormalTok{python nodes}\OperatorTok{/}\NormalTok{minimise.py }\OperatorTok{--}\NormalTok{steps}\OperatorTok{=}\DecValTok{1000} \OperatorTok{--}\NormalTok{files inputs}\OperatorTok{/}\NormalTok{ala.}\OperatorTok{*}
\end{Highlighting}
\end{Shaded}

We should find that the minimised molecular system has been written to
the working directory.

\begin{Shaded}
\begin{Highlighting}[]
\NormalTok{ls minimised.}\OperatorTok{*}
\end{Highlighting}
\end{Shaded}

\begin{verbatim}
minimised.prm7  minimised.rst7
\end{verbatim}

Note that the files have been written in the same format as the original
molecular system, i.e.~AMBER.

We have also provided some GROMACS format input files.

\begin{Shaded}
\begin{Highlighting}[]
\NormalTok{ls inputs}\OperatorTok{/}\NormalTok{kigaki.}\OperatorTok{*}
\end{Highlighting}
\end{Shaded}

\begin{verbatim}
inputs/kigaki.gro  inputs/kigaki.top
\end{verbatim}

Let's now run the node using these files as input. This is a larger
system so the minimisation will take a little longer.

\begin{Shaded}
\begin{Highlighting}[]
\NormalTok{python nodes}\OperatorTok{/}\NormalTok{minimise.py }\OperatorTok{--}\NormalTok{steps}\OperatorTok{=}\DecValTok{1000} \OperatorTok{--}\NormalTok{files inputs}\OperatorTok{/}\NormalTok{kigaki.}\OperatorTok{*}
\end{Highlighting}
\end{Shaded}

There should now be two additional GROMACS format output files in the
working directory.

\begin{Shaded}
\begin{Highlighting}[]
\NormalTok{ls minimised.}\OperatorTok{*}
\end{Highlighting}
\end{Shaded}

\begin{verbatim}
minimised.gro  minimised.prm7  minimised.rst7  minimised.top
\end{verbatim}

\hypertarget{running-nodes-from-within-biosimspace}{%
\subsection{Running nodes from within
BioSimSpace}\label{running-nodes-from-within-biosimspace}}

BioSimSpace also provides functionality for running nodes internally.
This allows you to call a node from within a script, thereby using
existing nodes as building blocks for more complicated workflows. To
activate nodes you can point BioSimSpace to a directory in which they
are contained. As such, you can maintain your own internal nodes and
have them available users when needed.

For example.

\begin{Shaded}
\begin{Highlighting}[]
\ImportTok{import}\NormalTok{ BioSimSpace }\ImportTok{as}\NormalTok{ BSS}

\NormalTok{BSS.Node.setNodeDirectory(}\StringTok{"nodes"}\NormalTok{)}
\NormalTok{BSS.Node.}\BuiltInTok{list}\NormalTok{()}
\end{Highlighting}
\end{Shaded}

\begin{verbatim}
['equilibrate', 'minimise', 'parameterise', 'solvate']
\end{verbatim}

To get information about a particular node we can pass its name to the
help function:

\begin{Shaded}
\begin{Highlighting}[]
\NormalTok{BSS.Node.}\BuiltInTok{help}\NormalTok{(}\StringTok{"minimise"}\NormalTok{)}
\end{Highlighting}
\end{Shaded}

\begin{verbatim}
usage: minimise.py [-h] [-c CONFIG] [-v [VERBOSE]] [--export-cwl [EXPORT_CWL]]
                   [--strict-file-naming [STRICT_FILE_NAMING]] --files FILES
                   [FILES ...] [--steps STEPS]
                   [--engine {Amber,Gromacs,Namd,OpenMM,Somd,auto}]

A node to perform energy minimisation and save the minimised molecular
configuration to file.

Args that start with '--' (e.g. --arg) can also be set in a config file
(specified via -c). The config file uses YAML syntax and must represent a YAML
'mapping' (for details, see http://learn.getgrav.org/advanced/yaml). If an arg
is specified in more than one place, then commandline values override config
file values which override defaults.

Output:
  minimised: FileSet    The minimised molecular system.

Required arguments:
  --files FILES [FILES ...]
                        A set of molecular input files.

Optional arguments:
  -h, --help            Show this help message and exit.
  -c CONFIG, --config CONFIG
                        Path to configuration file.
  -v [VERBOSE], --verbose [VERBOSE]
                        Print verbose error messages.
  --export-cwl [EXPORT_CWL]
                        Export Common Workflow Language (CWL) wrapper and exit.
  --strict-file-naming [STRICT_FILE_NAMING]
                        Enforce that the prefix of any file based output matches its name.
  --steps STEPS         The number of minimisation steps.
                          default=10000
                          min=0, max=1000000
  --engine {Amber,Gromacs,Namd,OpenMM,Somd,auto}
                        The molecular dynamics engine.
                          default=auto
\end{verbatim}

To execute a node we use the \texttt{run} function. This takes a
dictionary of input values and returns another dictionary containing the
outputs. Let's generate a valid input dictionary:

\begin{Shaded}
\begin{Highlighting}[]
\BuiltInTok{input} \OperatorTok{=}\NormalTok{ \{}\StringTok{"files"}\NormalTok{ : [}\StringTok{"inputs/ala.crd"}\NormalTok{, }\StringTok{"inputs/ala.top"}\NormalTok{],}
         \StringTok{"steps"}\NormalTok{ : }\DecValTok{1000}
\NormalTok{        \}}
\end{Highlighting}
\end{Shaded}

We can now run the \texttt{minimise} node, passing the dictionary from
above:

\begin{Shaded}
\begin{Highlighting}[]
\NormalTok{output }\OperatorTok{=}\NormalTok{ BSS.Node.run(}\StringTok{"minimise"}\NormalTok{, }\BuiltInTok{input}\NormalTok{)}
\end{Highlighting}
\end{Shaded}

Finally, let's print the output dictionary to see the result of running
the node:

\begin{Shaded}
\begin{Highlighting}[]
\BuiltInTok{print}\NormalTok{(output)}
\end{Highlighting}
\end{Shaded}

\begin{verbatim}
{'minimised': ['/home/lester/Code/BioSimSpaceTutorials/01_introduction/minimised.prm7', '/home/lester/Code/BioSimSpaceTutorials/01_introduction/minimised.rst7']}
\end{verbatim}

BioSimSpace nodes can also autogenerate their own
\href{https://www.commonwl.org/}{Common Workflow Language} (CWL) tool
wrappers, allowing them to be plugged into any workflow engine that
supports the standard. To generate a wrapper, simply pass the
\texttt{-\/-export-cwl} argument when running the node, e.g.:

\begin{Shaded}
\begin{Highlighting}[]
\NormalTok{python nodes}\OperatorTok{/}\NormalTok{equilibrate.py }\OperatorTok{--}\NormalTok{export}\OperatorTok{-}\NormalTok{cwl}
\end{Highlighting}
\end{Shaded}

Let's examine the wrapper:

\begin{Shaded}
\begin{Highlighting}[]
\NormalTok{cat nodes}\OperatorTok{/}\NormalTok{equilibrate.cwl}
\end{Highlighting}
\end{Shaded}

\begin{verbatim}
cwlVersion: v1.0
class: CommandLineTool
baseCommand: ["/home/lester/.conda/envs/biosimspace-dev/bin/python", "/home/lester/Code/BioSimSpaceTutorials/01_introduction/nodes/equilibrate.py", "--strict-file-naming"]

inputs:
  files:
    type:
      - type: array
        items: File
    inputBinding:
      prefix: --files
      separate: true

  runtime:
    type: string?
    default: 0.02 nanosecond
    inputBinding:
      prefix: --runtime
      separate: true

  temperature_start:
    type: string?
    default: 0.0 kelvin
    inputBinding:
      prefix: --temperature_start
      separate: true

  temperature_end:
    type: string?
    default: 300.0 kelvin
    inputBinding:
      prefix: --temperature_end
      separate: true

  restraint:
    type: string?
    default: none
    inputBinding:
      prefix: --restraint
      separate: true

  report_interval:
    type: int?
    default: 100
    inputBinding:
      prefix: --report_interval
      separate: true

  restart_interval:
    type: int?
    default: 500
    inputBinding:
      prefix: --restart_interval
      separate: true

  engine:
    type: string?
    default: auto
    inputBinding:
      prefix: --engine
      separate: true

outputs:
  equilibrated:
    type:
      type: array
      items: File
    outputBinding:
      glob: "equilibrated.*"
  trajectory:
    type: File
    outputBinding:
      glob: "trajectory.*"
\end{verbatim}

As a simple example of chaining BioSimSpace nodes in a command-line
workflow, consider the following script:

\begin{Shaded}
\begin{Highlighting}[]
\CommentTok{#!/usr/bin/env bash}
\CommentTok{# scripts/workflow.sh}

\CommentTok{# Exit immediately on error.}
\KeywordTok{set} \ExtensionTok{-e}

\BuiltInTok{echo} \StringTok{"Parameterising..."}
\ExtensionTok{python}\NormalTok{ nodes/parameterise.py --pdb inputs/methanol.pdb --forcefield gaff}

\BuiltInTok{echo} \StringTok{"Solvating..."}
\ExtensionTok{python}\NormalTok{ nodes/solvate.py --files parameterised.* --water_model tip3p}

\BuiltInTok{echo} \StringTok{"Minimising..."}
\ExtensionTok{python}\NormalTok{ nodes/minimise.py --files solvated.* --steps 1000}

\BuiltInTok{echo} \StringTok{"Equilibrating..."}
\ExtensionTok{python}\NormalTok{ nodes/equilibrate.py --files minimised.* --restraint heavy}

\BuiltInTok{echo} \StringTok{"Done!"}
\end{Highlighting}
\end{Shaded}

Starting from a PDB topology, this script calls each of the nodes in
sequence, passing the output of one as the input to the next. The output
of the final node is a set of files representing the equlibrated
molecular system, as well as a trajectory and PDB file that can be
visualised with, e.g.~the
\href{https://www.ks.uiuc.edu/Research/vmd/}{Visual Molecular Dynamics}
(VMD) program.

Let's run the workflow:

\begin{Shaded}
\begin{Highlighting}[]
\NormalTok{bash scripts}\OperatorTok{/}\NormalTok{workflow.sh}
\end{Highlighting}
\end{Shaded}

\begin{verbatim}
Parameterising...
Solvating...
Minimising...
Equilibrating...
Done!
\end{verbatim}
