\hypertarget{steered-molecular-dynamics}{%
\section{Steered molecular dynamics}\label{steered-molecular-dynamics}}

Author: Adele Hardie

Email: adele.hardie@ed.ac.uk

\hypertarget{requirements}{%
\paragraph{Requirements:}\label{requirements}}

\begin{itemize}
\tightlist
\item
  BioSimSpace
\item
  AMBER or GROMACS compiled with PLUMED
\item
  An equilibrated starting system
\item
  A target structure
\end{itemize}

This tutorial covers how to set up and run steered MD (sMD) simulations
with BioSimSpace. In this example, the sMD trajectory is used to
obtained seeds for equilibirum MD simulations for building a Markov
State Model (MSM). The overall protocol consists of 3 main steps:

\hypertarget{set-up-and-run-smd}{%
\subsubsection{1. Set up and run sMD}\label{set-up-and-run-smd}}

An example of how to set up and run sMD with BioSimSpace is provided in
\href{01_setup_sMD.md}{01\_setup\_sMD}. It covers some background for
sMD, creating a steering protocol that uses RMSD as the collective
variable (CV) for steering. A \href{01_setup_sMD_AMBER.ipynb}{notebook
version} exists as well. Additionally, a \href{01_run_sMD.py}{python
script} that can be run from command line is provided. Note that it is
specific to using RMSD of all heavy atoms of specified residues as the
CV, but it can be easily modified to work with other CVs
\href{https://biosimspace.org/api/index_Metadynamics_CollectiveVariable.html}{available
in BioSimSpace}. Example \href{01_run_sMD_LSF.sh}{LSF} and
\href{01_run_sMD_slurm.sh}{slurm} submission scripts are available.

\hypertarget{analysing-trajectory-and-seeded-md}{%
\subsubsection{2. Analysing trajectory and seeded
MD}\label{analysing-trajectory-and-seeded-md}}

Once an sMD run is complete, the trajectory can be analysed and
snapshots for seeded MD extracted
\href{02_trajectory_analysis.ipynb}{here}. In this case we only look at
the \texttt{COLVAR} file produced by PLUMED, but, depending on the
system, other criteria might be important to decide whether the sMD run
was successful. After the snapshots are extracted, they are used as
starting points for equilibrium MD simulations, the basics of which are
covered in the \href{../01_introduction}{introduction}. Note that the
\href{02_run_seededMD.py}{python script} is specific to the example
system of PTP1B due to additional parameters required, but the
\texttt{load\_system()} function can be easily modified to work woth
other systems. Since this step is very computationally intensive, it is
recommended to do this on an HPC system. An example array job
\href{02_run_seeded_MD_LSF.sh}{submission script} is available.

\hypertarget{build-the-msm}{%
\subsubsection{3. Build the MSM}\label{build-the-msm}}

After all the data for the MSM is gathered via seeded MD simulations, it
is time to build it. Here we use
\href{http://emma-project.org/latest/}{PyEMMA}, a python library with
their own
\href{http://emma-project.org/latest/tutorial.html}{tutorials}. There is
a lot to MSMs that could span a few workshops on its own, but for
illustration we provide \href{03_msm.md}{results} for PTP1B. For those
who are curious, a \href{03_msm_full.ipynb}{full notebook} describes
exactly how that particular MSM was built, but there are many ways of
doing it and so we refer you to PyEMMA's website for more detail.
