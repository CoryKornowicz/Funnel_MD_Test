Author: Adele Hardie

Email: adele.hardie@ed.ac.uk

\hypertarget{requirements}{%
\paragraph{Requirements:}\label{requirements}}

\begin{itemize}
\tightlist
\item
  pyemma
\item
  A number of MD trajectories
\end{itemize}

\hypertarget{markov-state-models}{%
\subsection{Markov State Models}\label{markov-state-models}}

Now that all of the required seeded MD simulations have run, we can use
the trajectory data to construct a Markov State Model (MSM).
\href{http://emma-project.org/latest/}{PyEMMA} is a python library for
this, and has their own
\href{http://emma-project.org/latest/tutorial.html}{tutorials}. There is
a lot to MSMs that could span a few workshops on its own, but an
\href{03_msm_full.ipynb}{example notebook} is available. For now, here
is the result of one build using the data obtained through the methods
covered previously:

The model indicates that this particular way of modelling PTP1B with
peptide substrate results in catalytically active conformations 2\% of
the time. Returning to the idea of allosteric inhibition, if a second
model, built for a system including an allosteric binder of interest,
showed a decrease in active conformation probability, it would suggest
that this binder has inhibition potential.
