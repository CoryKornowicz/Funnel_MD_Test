%%%%%%%%%%%%%%%%%%%%%%%%%%%%%%%%%%%%%%%%%%%%%%%%%%%%%%%%%%%%
%%% LIVECOMS ARTICLE TEMPLATE FOR TRAINING ARTICLE
%%% ADAPTED FROM ELIFE ARTICLE TEMPLATE (8/10/2017)
%%%%%%%%%%%%%%%%%%%%%%%%%%%%%%%%%%%%%%%%%%%%%%%%%%%%%%%%%%%%
%%% PREAMBLE
\documentclass[9pt,training]{livecoms}
% Use the 'onehalfspacing' option for 1.5 line spacing
% Use the 'doublespacing' option for 2.0 line spacing
% Use the 'lineno' option for adding line numbers.
% Use the "ASAPversion' option following article acceptance to add the DOI and relevant dates to the document footer.
% Use the 'pubversion' option for adding the citation and publication information to the document footer, when the LiveCoMS issue is finalized.
% The 'training' option for indicates that this is a training article.
% Omit the bestpractices option to remove the marking as a LiveCoMS paper.
% Please note that these options may affect formatting.

\usepackage{lipsum} % Required to insert dummy text
\usepackage[version=4]{mhchem}
\usepackage{siunitx}
\DeclareSIUnit\Molar{M}
\usepackage[italic]{mathastext}
\graphicspath{{figures/}}

%%%%%%%% USER PACKAGES
\usepackage{markdown}
\usepackage{minted}
\usepackage{tcolorbox}
\tcbuselibrary{minted, breakable, skins}
\definecolor{bg}{rgb}{244, 233, 228}
\usepackage{listings}
\lstset{
	basicstyle=\ttfamily,
	commentstyle={},
	breakatwhitespace=true,
	breaklines=true,
	language=bash
}

\newtcblisting{pythoncode}[2][]{    
  listing engine=minted,    
  breakable,   
  colback=bg,    
  listing only,    
  minted style=VS,    
  minted language=python,    
  minted options={texcl=true,#1},    
  #2,
}  
%%%%%%%%%%%%%%%%%%%%%%%%%%%%%%%%%%%%%%%%%%%%%%%%%%%%%%%%%%%%
%%% IMPORTANT USER CONFIGURATION
%%%%%%%%%%%%%%%%%%%%%%%%%%%%%%%%%%%%%%%%%%%%%%%%%%%%%%%%%%%%

\newcommand{\versionnumber}{1.3}  % you should update the minor version number in preprints and major version number of submissions.
\newcommand{\githubrepository}{\url{https://github.com/myaccount/homegithubrepository}}  %this should be the main github repository for this article

%%%%%%%%%%%%%%%%%%%%%%%%%%%%%%%%%%%%%%%%%%%%%%%%%%%%%%%%%%%%
%%% ARTICLE SETUP
%%%%%%%%%%%%%%%%%%%%%%%%%%%%%%%%%%%%%%%%%%%%%%%%%%%%%%%%%%%%
\title{This is the title [Article v\versionnumber]}

\author[1*]{Firstname Middlename Surname}
\author[1,2\authfn{1}\authfn{3}]{Firstname Middlename Familyname}
\author[2\authfn{1}\authfn{4}]{Firstname Initials Surname}
\author[2*]{Firstname Surname}
\affil[1]{Institution 1}
\affil[2]{Institution 2}

\corr{email1@example.com}{FMS}  % Correspondence emails.  FMS and FS are the appropriate authors initials.
\corr{email2@example.com}{FS}

\orcid{Author 1 name}{AAAA-BBBB-CCCC-DDDD}
\orcid{Author 2 name}{EEEE-FFFF-GGGG-HHHH}

\contrib[\authfn{1}]{These authors contributed equally to this work}
\contrib[\authfn{2}]{These authors also contributed equally to this work}

\presentadd[\authfn{3}]{Department, Institute, Country}
\presentadd[\authfn{4}]{Department, Institute, Country}

\blurb{This LiveCoMS document is maintained online on GitHub at \githubrepository; to provide feedback, suggestions, or help improve it, please visit the GitHub repository and participate via the issue tracker.}

%%%%%%%%%%%%%%%%%%%%%%%%%%%%%%%%%%%%%%%%%%%%%%%%%%%%%%%%%%%%
%%% PUBLICATION INFORMATION
%%% Fill out these parameters when available
%%% These are used when the "pubversion" option is invoked
%%%%%%%%%%%%%%%%%%%%%%%%%%%%%%%%%%%%%%%%%%%%%%%%%%%%%%%%%%%%
\pubDOI{10.XXXX/YYYYYYY}
\pubvolume{<volume>}
\pubissue{<issue>}
\pubyear{<year>}
\articlenum{<number>}
\datereceived{Day Month Year}
\dateaccepted{Day Month Year}

%%%%%%%%%%%%%%%%%%%%%%%%%%%%%%%%%%%%%%%%%%%%%%%%%%%%%%%%%%%%
%%% ARTICLE START
%%%%%%%%%%%%%%%%%%%%%%%%%%%%%%%%%%%%%%%%%%%%%%%%%%%%%%%%%%%%

\begin{document}

\begin{frontmatter}
\maketitle

\begin{abstract}
This particular document provides a skeleton illustrating key sections for a Training article.
Please see the sample \texttt{sample-document.tex} in \url{github.com/livecomsjournal/article_templates/templates} for additional information on and examples of using the LiveCoMS LaTeX class.
Here we also assume familiarity with LaTeX and knowledge of how to include figures, tables, etc.; if you want examples, see the sample just referenced.

In your work, in this particular slot, please provide an abstract of no more than 250 words.
Your abstract should explain the main contributions of your article, and should not contain any material that is not included in the main text.
Please note that your abstract, plus the authorship material following it, must not extend beyond the title page or modifications to the LaTeX class will likely be needed.
\end{abstract}

\end{frontmatter}



\begin{lstlisting}
$ command -arg1 -arg2 -arg3 -arg4 -arg5 -arg6 > output.file 2> err.file
\end{lstlisting}

\begin{pythoncode}
\\
import numpy as np
    
def incmatrix(genl1,genl2):
    M = []
    m = len(genl1)  # comment 
    for i in m:
        M.append(i)
    
    return M
\end{pythoncode}


{\scriptsize
\begin{lstlisting}[columns=flexible]

%VERSION  VERSION_STAMP = V0001.000  DATE = 06/30/15  11:44:23                  
%FLAG TITLE                                                                     
%FORMAT(20a4)                                                                   
ACE                                                                             
%FLAG POINTERS                                                                  
%FORMAT(10I8)                                                                   
    1912       9    1902       9      25      11      43      24       0       0
    2619     633       9      11      24      13      21      20      10       1
       0       0       0       0       0       0       0       1      10       0
       0
%FLAG ATOM_NAME                                                                 
%FORMAT(20a4)                                                                   
HH31CH3 HH32HH33C   O   N   H   CA  HA  CB  HB1 HB2 HB3 C   O   N   H   CH3 HH31
HH32HH33O   H1  H2  O   H1  H2  O   H1  H2  O   H1  H2  O   H1  H2  O   H1  H2  
O   H1  H2  O   H1  H2  O   H1  H2  O   H1  H2  O   H1  H2  O   H1  H2  O   H1  
H2  O   H1  H2  O   H1  H2  O   H1  H2  O   H1  H2  O   H1  H2  O   H1  H2  O   
H1  H2  O   H1  H2  O   H1  H2  O   H1  H2  O   H1  H2  O   H1  H2  O   H1  H2  
O   H1  H2  O   H1  H2  O   H1  H2  O   H1  H2  O   H1  H2  O   H1  H2  O   H1  
H2  O   H1  H2  O   H1  H2  O   H1  H2  O   H1  H2  O   H1  H2  O   H1  H2  O   
H1  H2  O   H1  H2  O   H1  H2  O   H1  H2  O   H1  H2  O   H1  H2  O   H1  H2 
\end{lstlisting}
}
\begin{lstlisting}[columns=flexible]

DISTANCERESSPEC
# DISH  DISC
  0.1   0.153
# i     j  k  l  type i    j  k  l  type r0     w0    rah
  1208  0  0  0  0    309  0  0  0  0    0.22   1.0   0
  1208  0  0  0  0    321  0  0  0  0    0.235  1.0   0
  1208  0  0  0  0    33   0  0  0  0    0.246  1.0   0
  120   0  0  0  0    489  0  0  0  0    0.255  1.0   0
  1208  0  0  0  0    490  0  0  0  0    0.248  1.0   0
END
\end{lstlisting}

\subsection{Steered Dynamics}

\subsection{Free Energy Perturbation}

%\appendix


\end{document}